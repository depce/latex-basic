\documentclass{article}
\usepackage{ctex}

\begin{document}

\title{创建目录}
\author{芦佳兴}
\tableofcontents
\maketitle
    \section{引言}
    新闻 [1]  ,也叫消息,是指通过报纸、电台、广播、电视台等媒体途径所传播信息的 [2]  一种称谓。
    是记录社会、传播信息、反映时代的一种文体。新闻概念有广义与狭义之分,就其广义而言,
    除了发表于报刊、广播、互联网、电视上的评论与专文外的常用文本都属于新闻之列,包括消息、通讯、特写、速写(
    有的将速写纳入特写之列)等等 [3]  ,狭义的新闻则专指消息,消息是用概括的叙述方式,以较简明扼要的文字,
    迅速及时地报道国内外新近发生的、有价值的事实,让别人了解。每则新闻一般包括标题、导语、主体、背景和结语五部分。
    前三者是主要部分,后二者是辅助部分。写法上主要是叙述,有时兼有议论、描写、评论等。新闻是包含海量资讯的新闻服务平台,
    真实反映每时每刻的重要事件。可以通过查看新闻事件、热点话题、人物动态、产品资讯等,快速了解它们的最新进展。 [4]

    有的将速写纳入特写之列)等等 [3]  ,狭义的新闻则专指消息,消息是用概括的叙述方式,以较简明扼要的文字,
    \\迅速及时地报道国内外新近发生的、有价值的事实,让别人了解。每则新闻一般包括标题、导语、主体、背景和结语五部分。
  

    \section{实验方法}
    \section{实验结果}
    \subsection{数据}
    \subsection{分析}
    \subsection{图表}
    \subsubsection{实验条件}
    \subsubsection{实验过程}
    \section{结论}
    \section{致谢}

\end{document}

% % 可以进行跳转的章节目录
% \documentclass[UTF8]{ctexart}

% \usepackage{hyperref}
% % 要调整章节标题在目录页中的格式,可以用titletoc 宏包。该宏包的基本命令参数如下:

% % 格式:\titlecontents{标题层次}[左间距]{整体格式}{标题序号}{标题内容}{指引线和页码}[下间距]

% \begin{document}

% \tableofcontents

% \part{部分标题}
% %\chapter{章标题}这一章我们介绍这些内容。
% \section{节标题}这一节我们介绍这些内容。
% \subsection{小节标题}这一小节我们介绍这些内容。
% \subsubsection{子节标题}这一子节我们介绍这些内容。
% \paragraph{段标题}这一段我们介绍这些内容。
% \subparagraph{小段标题}这一小段我们介绍这些内容。
% \end{document}