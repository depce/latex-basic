% % 导言区
% \documentclass{ctexart}
% % \usepackage{ctex}
% \usepackage{graphicx}
% \graphicspath{{figures/},{pics/..}} % 图片在当前目录下的 figures 目录
% % 正文区
% \begin{document}
%     在\LaTeX{}中的插图
%     \begin{figure}
%         \centering
%         \caption{\TeX 系统的吉祥物----小狮子 }
%         \includegraphics[scale=0.3]{123.jpg}
%     \end{figure}

%     在\LaTeX{}中的表格
%     \begin{table}
%         \centering
%         \caption{考试成绩表}
%         \begin{tabular}{|r||r|r|r|cr|}
%             \hline
%           姓名 & 语文 & 数学 & 英语 & 备注 & \\
%           \hline \hline
%           张三 & 87 & 100 & 95 & 优秀 & \\
%           \hline
%           李四 & 45 & 56 & 23 & 补考另行通知 & \\
%           \hline
%           王二 & 23 & 89 & 87 & &\\   
%           \hline
%         \end{tabular}
%     \end{table}
% \end{document}

\documentclass[UTF8]{ctexart}

\begin{document}

\begin{tabular}{|l|c|r|}
  \hline
  % after \\: \hline or \cline{col1-col2} \cline{col3-col4} ...
  左列 & 中列 & 右列 \\
  \hline
  2行1列 & 2行2列 & 2行3列 \\
  \hline
  3行1列 & 3行2列 & 3行3列 \\
  \hline
  4行1列 & 4行2列 & 4行3列 \\
  \hline


  \\
  % 跨越列 \multicolumn 命令的第一个参数指明要横跨的列数,第二个参数指明对齐和边框线,第三个参数指明该单元格的内容
  \hline
  % after \\: \hline or \cline{col1-col2} \cline{col3-col4} ...
  左列 & 中列 & 右列 \\
  \hline
  2行1列 & 2行2列 & 2行3列 \\
  \hline
  \multicolumn{2}{|c|}{跨越2015} & 3行3列 \\
  \hline
  4行1列 & 4行2列 & 4行3列 \\
  \hline
\end{tabular}

\end{document}