% \begin{abstract}         %摘要部分
%     \small\centering{俗话说,苦尽甘来,乐极生悲,其实痛苦与快乐,就象一对生死怨家,总是势不两立的;又象是一对孪生兄弟,如此亲密。所以痛苦过后,随之而来的,就是幸福、快乐,就好比不经历风雨,怎么能见彩虹,不经一番严寒彻骨,又怎能得到梅花的扑鼻香味呢?所谓苦尽甘来,也就有了越王勾践卧薪尝胆之后的国强民盛。而快乐过后,痛苦也随之而来,所谓乐极生悲,于是也就有了范进中举后的发疯。
%     有句俗语说,三十年河东,三十年河西;还有句俗语说,风水轮流转,明年到我家;(励志歌曲)由此可见,世间的一切,并没有一个定数,这包括了痛苦与快乐。所以,也就没有必要,因为短暂的快乐而喜形于色。得意忘形,甚至沉迷于内,更没必要为一时的痛苦而垂头丧气,由此而变得意志浮沉。
    
%     世人不明,总是不断地去追求快乐,追求幸福,却不愿意面对痛苦。没钱的希望发财,有钱的希望更多,有名的希望更响,当官的希望官做的更大,而一旦当这些失去时,巨大的失落,就成了一种剜心的痛苦。而过度的追求快乐时,那已经不是一种快乐,而是一种不断膨胀的欲望,当这种欲望冲昏了头脑并占据了自己的思想之时,最终会被这种欲望埋葬了自己。
%     }
% \end{abstract}

\begin{abstract}
    \setlength{\parindent}{2em}%%首段缩进长度
    \setlength{\baselineskip}{1.8em}%%基本行距
    \setlength{\parskip}{1ex}%%段落间距
    \setlength{\abovedisplayskip}{3pt} %% 3pt 与顶部距离 个人觉得稍妥,可自行设置
    \setlength{\belowdisplayskip}{3pt} %% 3pt 与底部距离 个人觉得稍妥,可自行设置
     本文以《冰与火之歌》中的世界为背景,从生物学和地理学两个方面对龙的生活进行分析,探索了冰与火之歌中三条龙的生长、能量摄入、能量支出等生理状态与生存面积、气候对龙的影响等地理状态。
    \par %%另起一段落
    针对问题一......
    %% 中途省略了部分文字
    \par
    最后我们对模型进行了灵敏度分析,分析了温度对于龙的代谢速率的影响情况,结果如图所示,结果显示温度对龙的代谢速率的影响远小于体重对于代谢速率的影响,我们模型的前提假设是成立的,模型是具有鲁棒性的。
    \par
    \textbf{关键字:}Logistic 阻滞增长模型;层次分析法;能量收支平衡模型;生态足迹模型;气候综合舒适度指数
\end{abstract}